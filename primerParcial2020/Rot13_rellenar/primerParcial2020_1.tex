\documentclass[12pt]{article} % Clase de documento: artículo y tamaño de letra
\usepackage[utf8]{inputenc} % Escritura en castellano con acentos
\usepackage[T1]{fontenc} % Escritura en castellano con acentos
\usepackage{calligra} 
\usepackage{listings}
\usepackage{pslatex}  % Fuente de letras
\usepackage{graphicx}
\usepackage{array}
\usepackage[margin=2.5cm]{geometry}
\usepackage{caption}
\usepackage{txfonts}
\usepackage{xcolor}
\usepackage{fancyhdr}
\usepackage[spanish,activeacute]{babel}
\usepackage{graphicx}
\usepackage{float}
\usepackage{multicol}
\usepackage{color}
\usepackage{times}
\usepackage{tikz}
\usepackage{verbatim}
\usepackage{mdwlist}
\usetikzlibrary{chains,fit,shapes}
\usepackage{array}
\usepackage{caption}

\usetikzlibrary{chains,fit,shapes,arrows,calc,shapes,decorations.pathreplacing}
\usetikzlibrary{through}


\DeclareCaptionFont{white}{\color{white}}
\DeclareCaptionFormat{listing}{\colorbox[cmyk]{0.43, 0.35, 0.35,0.01}{\parbox{\textwidth}{\hspace{15pt}#1#2#3}}}

%Paquetes para hacer tablas lindas...
% \usepackage{amsmath,amssymb,amsfonts,latexsym,stmaryrd}
\usepackage{tabularx}
\usepackage{colortbl}
\usepackage{tikz}
\usepackage{shadow}
\usepackage{fancybox}
\usepackage{url}
\usepackage[hidelinks]{hyperref}
\usepackage{subfigure}
\usepackage{multirow}


\definecolor{OliveGreen}{RGB}{2,80,1}
\definecolor{LigthOrange}{RGB}{255,255,200}
\definecolor{gray97}{gray}{.97}
\definecolor{Gray}{RGB}{171,174,178}

\newcommand{\instr}[1]{{\sffamily{\small{\textsl{\textbf{#1}}}}}}
\newcommand{\code}[1]{{\sffamily{\smal\textheight = 24cm \textwidth = 16cm \topmargin = -1cm \oddsidemargin= 0cm {\textsl{#1}}}}}

\lstdefinelanguage {x86nasm}
{morekeywords={resb,resw,resd,resq,endstruc,at,istruc,iend}}

\pagestyle{fancy}
\headheight=50pt %para cambiar el tamaño del encabezado
\fancyhead[L]    %la "L" indica a la izquierda
{	
 \begin{minipage}{2cm}
  \includegraphics[width=1.3\textwidth]{Logo-UTN-BA.jpeg}
 \end{minipage}	
 \begin{minipage}{10.9cm}
\begin{center}
   \Large
   {
     \textsf
     {
       \calligra{Universidad Tecnológica Nacional\\ Facultad Regional Buenos Aires \\ Departamento de Ingeniería Electrónica} 
     }
   }
\end{center}
\end{minipage}
}

\fancyhead[R] %la "R" indica a la derecha
{
  \begin{minipage}{3cm}
   \small
   {
     \emph{\textbf{Informática I}} \\ \emph{4 de Julio de 2020} \\ \emph{Primer Parcial}\\ \emph{Curso R1021}  
   }
  \end{minipage}}

\definecolor{OliveGreen}{RGB}{2,80,1}
\definecolor{LigthOrange}{RGB}{255,255,200}
\definecolor{gray97}{gray}{.97}



\begin{document} % Inicio del documento
\newpage
%Cambia de página, el texto después de este comando aparecerá en la siguiente página en adelante.
\noindent
  \begin{center}
   \begin{tabular}{| c | c | c |}
    \hline
     Apellido y Nombres \hspace{8cm} &  Legajo & {Calificación} \\ \hline 
      &	& \\ \hline
   \end{tabular}	
  \end{center}

\noindent
%{\bf Condiciones de aprobación:} \\
%Bla
%Bla
%Bla para el {\bf primerParcial}.\\

%%%%%%%%%%%%%%%%%%%%%%%%%%%%%%%%%%%%%%%%%%%%%%%%%%%%%%%%%%%%%%%%%%%%%%%%%%%%%%%%%%%%%%%%%%%%%%%%%%%%%%%%%%%%%%%
%%%%%%%%%%%%%%%%%%%%%%%%%%%%%%%%%%%%%%%%%%%%%%%%%%%%%%%%%%%%%%%%%%%%%%%%%%%%%%%%%%%%%%%%%%%%%%%%%%%%%%%%%%%%%%%
% Comienzo del tema del examen
%%%%%%%%%%%%%%%%%%%%%%%%%%%%%%%%%%%%%%%%%%%%%%%%%%%%%%%%%%%%%%%%%%%%%%%%%%%%%%%%%%%%%%%%%%%%%%%%%%%%%%%%%%%%%%%
%%%%%%%%%%%%%%%%%%%%%%%%%%%%%%%%%%%%%%%%%%%%%%%%%%%%%%%%%%%%%%%%%%%%%%%%%%%%%%%%%%%%%%%%%%%%%%%%%%%%%%%%%%%%%%%

\lstset{
	frame=Ltb,
	framerule=0pt,
	aboveskip=0.5cm,
	framextopmargin=3pt,
	framexbottommargin=3pt,
	framexleftmargin=0.4cm,
	framesep=0pt,
	rulesep=.4pt,
	backgroundcolor=\color{gray97},
	rulesepcolor=\color{black},
 	language=C,
	captionpos=b,
	tabsize=3,
	frame=lines,
	keywordstyle=\color{blue},
	commentstyle=\color{Gray},
	stringstyle=\color{red},
	numbers=left,
	numberstyle=\tiny,
	numbersep=5pt,
	breaklines=true,
	showstringspaces=false,
	basicstyle=\ttfamily\scriptsize,
	emph={label},
	framerule=0pt,
}

\begin{enumerate}
\item Implemente una función que contenga el algoritmo de cifrado {\bf ROT13}. \\
El algoritmo consiste en sustituir cada letra por una que se encuentra trece ( 13 ) posiciones por delante o por detras según corresponda. \\
Por ejemplo la A se reemplaza por la N, la B por la O y así sucesivamente. Para las últimas trece letras la secuencia se invierte.\\
{\color{red}Se debe mostrar el mensaje original, el mensaje codificado y la cantidad de caracteres que fueron sustituidos}. También se debe utilizar la menor cantidad de memoria posible y el mensaje no puede exeder 30 caracteres.\\
 A continuación se muestra la tabla de las equivalencias entre las letras.
%\vspace{0.6cm}
\noindent
  \begin{center}
   \begin{tabular}{| c | c | c | c | c | c | c | c | c | c | c | c | c |}
    \hline
     A & B & C & D & E & F & G & H & I & J & K & L & M \\ \hline 
     N & O & P & Q & R & S & T & U & V & W & X & Y & Z \\ \hline 
   \end{tabular}	
  \end{center}

  \begin{center}
   \begin{tabular}{| c | c | c | c | c | c | c | c | c | c | c | c | c |}
    \hline
     a & b & c & d & e & f & g & h & i & j & k & l & m \\ \hline 
     n & o & p & q & r & s & t & u & v & w & x & y & z \\ \hline 
   \end{tabular}	
  \end{center}

Ejemplo:
\noindent
\vspace{-0.6cm}
  \begin{center}
   \begin{tabular}{| c | c | c | c | c | c | c | c | c | c | c |}
    \hline
     Texto sin cifrar & H & o & L & a &   & m & u & N & d & O  \\ \hline 
     Texto cifrado    & U & b & Y & n &   & z & h & A & q & B  \\ \hline 
   \end{tabular}	
  \end{center}
Prototipo de la función:
\begin{center}
{\color{blue}int} rot\_13 ({\color{blue}char} *dataPtr , {\color{blue}char} * dataPtrCodificada)
\end{center}
Donde:
\begin{itemize}
\item {\bf dataPtr:} es el puntero al mensaje a cifrar (string).
\item {\bf dataPtrCodificada:} es el puntero al mensaje cifrado (string) donde se colocará el mensaje cifrado.
\end{itemize}

Devuelve:
\begin{itemize}
\item {\bf -1} Si el mensaje contiene un caracter {\bf distinto} de una {\bf letra} o un {\bf espacio}.
\item Un {\bf número positivo} indicando la cantidad de caracteres convertidos sin tener en 
cuenta los espacios ni el {\bf $\setminus$0}
\end{itemize}
\end{enumerate}

\newpage

Donde se ve  {\color{red}(........)}  completar con el dato necesario.\\ 
 \begin{lstlisting}
#include <stdio.h>
#include <string.h>
#include <stdlib.h>
#define TAM 30

int rot_13(char* , char *);        

int main(void)
{
    char dataPtr[TAM], *dataPtrCodificada;        
    int devuelve,longitud,i;

    printf("Ingrese texto a codificar\n");
    fgets(dataPtr,TAM,stdin);
    for (i=0;dataPtr[i]!='\0';i++){     
    }
    dataPtr[i-1]='\0';                  
    longitud = strlen(dataPtr);         

    dataPtrCodificada = (char *) malloc ( (longitud+1) * sizeof(char)); 

    devuelve = rot_13( (.......) , (.......));         
    if (devuelve == -1){
        printf("el mensaje contiene un caracter distinto de una letra o un espacio\n");
    }else{
        printf("el mensaje %s fue codificado a %s\n",(.......), (.......));
        printf("la cantidad intercambiada fue de: %d\r\n",(.......));
    }
    free({(.......));
    return (0);
}
   \end{lstlisting}
 
\end{document} % Fin del documento.

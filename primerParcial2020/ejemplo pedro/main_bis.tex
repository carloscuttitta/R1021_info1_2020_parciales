\documentclass[12pt]{article} % Clase de documento: artículo y tamaño de letra
\usepackage[utf8]{inputenc} % Escritura en castellano con acentos
\usepackage{calligra} 
\usepackage{listings}
\usepackage[margin=2.5cm]{geometry}
\usepackage{fancyhdr}
\usepackage[spanish,activeacute]{babel}
\usepackage{tikz}
\usepackage{verbatim}

\pagestyle{fancy}
\headheight=50pt %para cambiar el tamaño del encabezado
\fancyhead[L]    %la "L" indica a la izquierda
{	
 \begin{minipage}{3.3cm}
  \includegraphics[width=1.0\textwidth]{Logo-UTN-BA.jpeg}
%  \includegraphics[width=0.6\textwidth]{UTNlogo.eps}
 \end{minipage}	
 \begin{minipage}{7.7cm}
   \normalsize
   {
     \textsf
     {
       \calligra{Universidad Tecnológica Nacional\\ Facultad Regional Buenos Aires \\ Departamento de {Ingeniería} {Electrónica}} 
     }
   }
 \end{minipage}
}

\fancyhead[R] %la "R" indica a la derecha
{
  \begin{minipage}{4.0cm}
   \small
   {
     \emph{\textbf{Informática I}} \\ \emph{11 de Julio de 2020} \\ \emph{Pedro Vassena}\\ \emph{Curso R1021}  
   }
  \end{minipage}}

\definecolor{OliveGreen}{RGB}{2,80,1}
\definecolor{LigthOrange}{RGB}{255,255,200}
\definecolor{gray97}{gray}{.97}

\begin{document} % Inicio del documento
\newpage
%Cambia de página, el texto después de este comando aparecerá en la siguiente página en adelante.

\noindent
  \begin{center}
   \begin{tabular}{| c | c | c |}
    \hline
     Apellido y Nombres \hspace{8cm} &  Legajo & {Calificación} \\ \hline 
      &	& \\ \hline
   \end{tabular}	
  \end{center}

\noindent
%Condiciones: \\

%%%%%%%%%%%%%%%%%%%%%%%%%%%%%%%%%%%%%%%%%%%%%%%%%%%%%%%%%%%%%%%%%%%%%%%%%%%%%%%%%%%%%%%%%%%%%%%%%%%%%%%%%%%%%%%
%%%%%%%%%%%%%%%%%%%%%%%%%%%%%%%%%%%%%%%%%%%%%%%%%%%%%%%%%%%%%%%%%%%%%%%%%%%%%%%%%%%%%%%%%%%%%%%%%%%%%%%%%%%%%%%
% Comienzo del tema del examen
%%%%%%%%%%%%%%%%%%%%%%%%%%%%%%%%%%%%%%%%%%%%%%%%%%%%%%%%%%%%%%%%%%%%%%%%%%%%%%%%%%%%%%%%%%%%%%%%%%%%%%%%%%%%%%%
%%%%%%%%%%%%%%%%%%%%%%%%%%%%%%%%%%%%%%%%%%%%%%%%%%%%%%%%%%%%%%%%%%%%%%%%%%%%%%%%%%%%%%%%%%%%%%%%%%%%%%%%%%%%%%%

\lstset{
	frame=Ltb,
	framerule=0pt,
	aboveskip=0.5cm,
	framextopmargin=3pt,
	framexbottommargin=3pt,
	framexleftmargin=0.4cm,
	framesep=0pt,
	rulesep=.4pt,
	backgroundcolor=\color{gray97},
	rulesepcolor=\color{black},
 	language=C,
	captionpos=b,
	tabsize=3,
	frame=lines,
	keywordstyle=\color{blue},
	commentstyle=\color{Gray},
	stringstyle=\color{red},
	numbers=left,
	numberstyle=\tiny,
	numbersep=5pt,
	breaklines=true,
	showstringspaces=false,
	basicstyle=\ttfamily\scriptsize,
	emph={label},
	framerule=0pt,
}
\begin{enumerate}
 \item Realizar una función que sume 2 bytes no signados con saturación.

Prototipo de la función:
\lstset{language=c,
	numbers=none,
	numberstyle=,
	numbersep=,
}
 \begin{lstlisting}
unsigned char suma_saturada(unsigned char a, unsigned char b);
\end{lstlisting}
Parámetros de entrada:

\lstset{language=c,
	numbers=none,
	numberstyle=,
	numbersep=,
}
 \begin{lstlisting}
unsigned char a: Byte no signado con el primer operando.
unsigned char b: Byte no signado con el segundo operando.
\end{lstlisting}

Retorno:

\lstset{language=c,
	numbers=none,
	numberstyle=,
	numbersep=,
}
 \begin{lstlisting}
unsigned char: Byte no signado con el resultado.
\end{lstlisting}

La suma saturada implica que si hay {\color{blue}overflow} (la suma de dos valores no signados da menor que un operando cualquiera) se debe devolver el valor
\lstset{language=c,
	numbers=none,
	numberstyle=,
	numbersep=,
}
 \begin{lstlisting}
UCHAR_MAX (+255).
\end{lstlisting}

Este define está definido en 
\lstset{language=c,
	numbers=none,
	numberstyle=,
	numbersep=,
}

 \begin{lstlisting}
#include<limits.h>
\end{lstlisting}

\item Realizar una función que sume 2 vectores de bytes no signados con saturación.

Prototipo de la función:
\lstset{language=c,
	numbers=none,
	numberstyle=,
	numbersep=,
}

 \begin{lstlisting}
unsigned char *suma_saturada_vector(unsigned char *vec1,unsigned char *vec2,unsigned int n);
\end{lstlisting}

Parametros de entrada:
\lstset{language=c,
	numbers=none,
	numberstyle=,
	numbersep=,
}

 \begin{lstlisting}
char *vec1: Puntero que apunta al primer elemento de un vector de bytes no signados de longitud n.
char *vec2: Puntero que apunta al primer elemento de otro vector de bytes no signados de longitud n.
unsigned int n: Longitud de los vectores a sumar.
\end{lstlisting}

Retorno:
\lstset{language=c,
	numbers=none,
	numberstyle=,
	numbersep=,
}

 \begin{lstlisting}
char *: Puntero que apunta al primer elemento de un vector de bytes no signados de longitud n, alojado dinamicamente, que contiene el resultado de la suma.
\end{lstlisting}

Utilizar la función del punto 1.\\
Utilizar memoria dinámica para la salida.

\end{enumerate}
 
\end{document} % Fin del documento.

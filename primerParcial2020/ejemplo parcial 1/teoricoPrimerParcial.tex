\documentclass[12pt]{article} % Clase de documento: artículo y tamaño de letra
%\usepackage[spanish]{babel} % Manejo de idiomas
\usepackage[utf8]{inputenc} % Escritura en castellano con acentos
\usepackage[T1]{fontenc} % Escritura en castellano con acentos
\usepackage{calligra} 
\usepackage{listings}
\usepackage{pslatex}  % Fuente de letras
\usepackage{graphicx}
\usepackage{array}
%\usepackage{times} % Fuente de letras
\usepackage[margin=2.5cm]{geometry}
%Coloca 2.5 cm de margenes superior, inferior,derecho e izquierdo.
\usepackage{caption}
\usepackage{txfonts}
\usepackage{xcolor}
\usepackage{fancyhdr}
\usepackage[spanish,activeacute]{babel}
\usepackage{float}
\usepackage{multicol}
\usepackage{color}
% \usepackage[T1]{fontenc}
\usepackage{times}
% \usepackage[lmargin=2cm]{geometry}
\usepackage{tikz}
\usepackage{verbatim}
\usepackage{mdwlist}

\usetikzlibrary{chains,fit,shapes,arrows,calc,shapes,decorations.pathreplacing}
\usetikzlibrary{through}

\DeclareCaptionFont{white}{\color{white}}
\DeclareCaptionFormat{listing}{\colorbox[cmyk]{0.43, 0.35, 0.35,0.01}{\parbox{\textwidth}{\hspace{15pt}#1#2#3}}}

%Paquetes para hacer tablas lindas...
% \usepackage{amsmath,amssymb,amsfonts,latexsym,stmaryrd}
\usepackage{tabularx}
\usepackage{colortbl}
\usepackage{shadow}
\usepackage{fancybox}
\usepackage{url}
\usepackage[hidelinks]{hyperref}
\usepackage{subfigure}
\usepackage{multirow}

\pagestyle{fancy}
\headheight=50pt %para cambiar el tamaño del encabezado
\fancyhead[L]    %la "L" indica a la izquierda
{	
 \begin{minipage}{3.3cm}
  \includegraphics[width=1.0\textwidth]{Logo-UTN-BA.jpeg}
%  \includegraphics[width=0.6\textwidth]{UTNlogo.eps}
 \end{minipage}	
 \begin{minipage}{7.7cm}
   \normalsize
   {
     \textsf
     {
       \calligra{Universidad Tecnológica Nacional\\ Facultad Regional Buenos Aires \\ Departamento de {Ingeniería} {Electrónica}} 
     }
   }
 \end{minipage}
}

\fancyhead[R] %la "R" indica a la derecha
{
  \begin{minipage}{4.0cm}
   \small
   {
     \emph{\textbf{Informática I}} \\ \emph{18 de Julio de 2020} \\ \emph{Primer parcial}\\ \emph{Curso R1021}  
   }
  \end{minipage}}

\definecolor{OliveGreen}{RGB}{2,80,1}
\definecolor{LigthOrange}{RGB}{255,255,200}
\definecolor{gray97}{gray}{.97}

\begin{document} % Inicio del documento
\newpage
%Cambia de página, el texto después de este comando aparecerá en la siguiente página en adelante.

\noindent
  \begin{center}
   \begin{tabular}{| c | c | c |}
    \hline
     Apellido y Nombres \hspace{8cm} &  Legajo & {Calificación} \\ \hline 
      &	& \\ \hline
   \end{tabular}	
  \end{center}

\noindent
%Condiciones: \\

%%%%%%%%%%%%%%%%%%%%%%%%%%%%%%%%%%%%%%%%%%%%%%%%%%%%%%%%%%%%%%%%%%%%%%%%%%%%%%%%%%%%%%%%%%%%%%%%%%%%%%%%%%%%%%%
%%%%%%%%%%%%%%%%%%%%%%%%%%%%%%%%%%%%%%%%%%%%%%%%%%%%%%%%%%%%%%%%%%%%%%%%%%%%%%%%%%%%%%%%%%%%%%%%%%%%%%%%%%%%%%%
% Comienzo del tema del examen
%%%%%%%%%%%%%%%%%%%%%%%%%%%%%%%%%%%%%%%%%%%%%%%%%%%%%%%%%%%%%%%%%%%%%%%%%%%%%%%%%%%%%%%%%%%%%%%%%%%%%%%%%%%%%%%
%%%%%%%%%%%%%%%%%%%%%%%%%%%%%%%%%%%%%%%%%%%%%%%%%%%%%%%%%%%%%%%%%%%%%%%%%%%%%%%%%%%%%%%%%%%%%%%%%%%%%%%%%%%%%%%

\lstset{
	frame=Ltb,
	framerule=0pt,
	aboveskip=0.5cm,
	framextopmargin=3pt,
	framexbottommargin=3pt,
	framexleftmargin=0.4cm,
	framesep=0pt,
	rulesep=.4pt,
	backgroundcolor=\color{gray97},
	rulesepcolor=\color{black},
 	language=C,
	captionpos=b,
	tabsize=3,
	frame=lines,
	keywordstyle=\color{blue},
	commentstyle=\color{Gray},
	stringstyle=\color{red},
	numbers=left,
	numberstyle=\tiny,
	numbersep=5pt,
	breaklines=true,
	showstringspaces=false,
	basicstyle=\ttfamily\scriptsize,
	emph={label},
	framerule=0pt,
}
\begin{enumerate}
\item Indique que diferencia hay entre pasar un dato a una función por valor y por referencia.

\item Explique el funcionamiento del siguiente programa. Indique que se imprime por stdout

\begin{lstlisting}
#include <stdio.h>

int main (void)
{
    unsigned char i;
    char j;

    for (i = 0; i < 128; i++) {
        printf ("%x\t", i);
    }
    puts ("");
    
    for (j = 0; j < 128; j++) {
        printf ("%x\t", j);
    }
    puts ("");
    return (0);
}
\end{lstlisting}

\end{enumerate}


 
\end{document} % Fin del documento.
 

  


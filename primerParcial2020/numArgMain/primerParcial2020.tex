\documentclass[12pt]{article} % Clase de documento: artículo y tamaño de letra
\usepackage[utf8]{inputenc} % Escritura en castellano con acentos
\usepackage[T1]{fontenc} % Escritura en castellano con acentos
\usepackage{calligra} 
\usepackage{listings}
\usepackage{pslatex}  % Fuente de letras
\usepackage{graphicx}
\usepackage{array}
\usepackage[margin=2.5cm]{geometry}
\usepackage{caption}
\usepackage{txfonts}
\usepackage{xcolor}
\usepackage{fancyhdr}
\usepackage[spanish,activeacute]{babel}
\usepackage{graphicx}
\usepackage{float}
\usepackage{multicol}
\usepackage{color}
\usepackage{times}
\usepackage{tikz}
\usepackage{verbatim}
\usepackage{mdwlist}
\usetikzlibrary{chains,fit,shapes}
\usepackage{array}
\usepackage{caption}

\usetikzlibrary{chains,fit,shapes,arrows,calc,shapes,decorations.pathreplacing}
\usetikzlibrary{through}


\DeclareCaptionFont{white}{\color{white}}
\DeclareCaptionFormat{listing}{\colorbox[cmyk]{0.43, 0.35, 0.35,0.01}{\parbox{\textwidth}{\hspace{15pt}#1#2#3}}}

%Paquetes para hacer tablas lindas...
% \usepackage{amsmath,amssymb,amsfonts,latexsym,stmaryrd}
\usepackage{tabularx}
\usepackage{colortbl}
\usepackage{tikz}
\usepackage{shadow}
\usepackage{fancybox}
\usepackage{url}
\usepackage[hidelinks]{hyperref}
\usepackage{subfigure}
\usepackage{multirow}


\definecolor{OliveGreen}{RGB}{2,80,1}
\definecolor{LigthOrange}{RGB}{255,255,200}
\definecolor{gray97}{gray}{.97}
\definecolor{Gray}{RGB}{171,174,178}

\newcommand{\instr}[1]{{\sffamily{\small{\textsl{\textbf{#1}}}}}}
\newcommand{\code}[1]{{\sffamily{\smal\textheight = 24cm \textwidth = 16cm \topmargin = -1cm \oddsidemargin= 0cm {\textsl{#1}}}}}

\lstdefinelanguage {x86nasm}
{morekeywords={resb,resw,resd,resq,endstruc,at,istruc,iend}}

\pagestyle{fancy}
\headheight=50pt %para cambiar el tamaño del encabezado
\fancyhead[L]    %la "L" indica a la izquierda
{	
 \begin{minipage}{2cm}
  \includegraphics[width=1.3\textwidth]{Logo-UTN-BA.jpeg}
 \end{minipage}	
 \begin{minipage}{10.9cm}
\begin{center}
   \Large
   {
     \textsf
     {
       \calligra{Universidad Tecnológica Nacional\\ Facultad Regional Buenos Aires \\ Departamento de Ingeniería Electrónica} 
     }
   }
\end{center}
\end{minipage}
}

\fancyhead[R] %la "R" indica a la derecha
{
  \begin{minipage}{3cm}
   \small
   {
     \emph{\textbf{Informática I}} \\ \emph{4 de Julio de 2020} \\ \emph{Primer Parcial}\\ \emph{Curso R1021}  
   }
  \end{minipage}}

\definecolor{OliveGreen}{RGB}{2,80,1}
\definecolor{LigthOrange}{RGB}{255,255,200}
\definecolor{gray97}{gray}{.97}



\begin{document} % Inicio del documento
\newpage
%Cambia de página, el texto después de este comando aparecerá en la siguiente página en adelante.
\noindent
  \begin{center}
   \begin{tabular}{| c | c | c |}
    \hline
     Apellido y Nombres \hspace{8cm} &  Legajo & {Calificación} \\ \hline 
      &	& \\ \hline
   \end{tabular}	
  \end{center}

\noindent
%{\bf Condiciones de aprobación:} \\
%Bla
%Bla
%Bla para el {\bf primerParcial}.\\

%%%%%%%%%%%%%%%%%%%%%%%%%%%%%%%%%%%%%%%%%%%%%%%%%%%%%%%%%%%%%%%%%%%%%%%%%%%%%%%%%%%%%%%%%%%%%%%%%%%%%%%%%%%%%%%
%%%%%%%%%%%%%%%%%%%%%%%%%%%%%%%%%%%%%%%%%%%%%%%%%%%%%%%%%%%%%%%%%%%%%%%%%%%%%%%%%%%%%%%%%%%%%%%%%%%%%%%%%%%%%%%
% Comienzo del tema del examen
%%%%%%%%%%%%%%%%%%%%%%%%%%%%%%%%%%%%%%%%%%%%%%%%%%%%%%%%%%%%%%%%%%%%%%%%%%%%%%%%%%%%%%%%%%%%%%%%%%%%%%%%%%%%%%%
%%%%%%%%%%%%%%%%%%%%%%%%%%%%%%%%%%%%%%%%%%%%%%%%%%%%%%%%%%%%%%%%%%%%%%%%%%%%%%%%%%%%%%%%%%%%%%%%%%%%%%%%%%%%%%%

\lstset{
	frame=Ltb,
	framerule=0pt,
	aboveskip=0.5cm,
	framextopmargin=3pt,
	framexbottommargin=3pt,
	framexleftmargin=0.4cm,
	framesep=0pt,
	rulesep=.4pt,
	backgroundcolor=\color{gray97},
	rulesepcolor=\color{black},
% 	language=C,
	captionpos=b,
	tabsize=3,
	frame=lines,
	keywordstyle=\color{blue},
	commentstyle=\color{Gray},
	stringstyle=\color{red},
	numbers=left,
	numberstyle=\tiny,
	numbersep=5pt,
	breaklines=true,
	showstringspaces=false,
	basicstyle=\ttfamily\scriptsize,
	emph={label},
	framerule=0pt,
}

\begin{enumerate}
\item Realizar un programa que por línea de comando reciba una cantidad indeterminada de argumentos. Los mismos serán considerados como términos de una serie del tipo entero.\\
Se deberá presentar por {\color{blue}stdout} las siguientes operaciones:
\begin{itemize}
\item Suma de todos los términos
\item Promedio de todos los términos
\item El mayor de todos
\item El menor de todos
\end{itemize}
Se tiene que validar los errores que considere pertinente terminado el programa con un código de error para cada caso. Si la finalización es exitosa, terminará con código de retorno {\color{blue}0}.
El programa deberá tener un solo punto de salida. 


\end{enumerate}
 
\end{document} % Fin del documento.

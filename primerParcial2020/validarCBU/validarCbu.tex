\documentclass[12pt]{article} % Clase de documento: artículo y tamaño de letra
 
% \usepackage[spanish]{babel} % Manejo de idiomas
\usepackage[utf8]{inputenc} % Escritura en castellano con acentos
\usepackage[T1]{fontenc} % Escritura en castellano con acentos
\usepackage{calligra} 
\usepackage{listings}
\usepackage{pslatex}  % Fuente de letras
\usepackage{graphicx}
\usepackage{array}
%\usepackage{times} % Fuente de letras
\usepackage[margin=2.5cm]{geometry}
%Coloca 2.5 cm de margenes superior, inferior,derecho e izquierdo.
\usepackage{caption}
\usepackage{txfonts}
\usepackage{xcolor}
\usepackage{fancyhdr}

\usepackage[spanish,activeacute]{babel}
\usepackage{float}
\usepackage{multicol}
\usepackage{color}
\usepackage{times}

\usepackage{tikz}
\usepackage{verbatim}
\usepackage{mdwlist}


\usetikzlibrary{chains,fit,shapes,arrows,calc,shapes,decorations.pathreplacing}
\usetikzlibrary{through}


\DeclareCaptionFont{white}{\color{white}}
\DeclareCaptionFormat{listing}{\colorbox[cmyk]{0.43, 0.35, 0.35,0.01}{\parbox{\textwidth}{\hspace{15pt}#1#2#3}}}

%Paquetes para hacer tablas lindas...
% \usepackage{amsmath,amssymb,amsfonts,latexsym,stmaryrd}
\usepackage{tabularx}
\usepackage{colortbl}
\usepackage{shadow}
\usepackage{fancybox}
\usepackage{url}
\usepackage[hidelinks]{hyperref}
\usepackage{subfigure}
\usepackage{multirow}


\definecolor{OliveGreen}{RGB}{2,80,1}
\definecolor{LigthOrange}{RGB}{255,255,200}
\definecolor{gray97}{gray}{.97}
\definecolor{Gray}{RGB}{171,174,178}

\newcommand{\instr}[1]{{\sffamily{\small{\textsl{\textbf{#1}}}}}}
\newcommand{\code}[1]{{\sffamily{\smal\textheight = 24cm \textwidth = 16cm \topmargin = -1cm \oddsidemargin= 0cm {\textsl{#1}}}}}

\lstdefinelanguage {x86nasm}
{morekeywords={resb,resw,resd,resq,endstruc,at,istruc,iend}}

\pagestyle{fancy}
\headheight=50pt %para cambiar el tamaño del encabezado
\fancyhead[L]    %la "L" indica a la izquierda
{	
 \begin{minipage}{3.3cm}
  \includegraphics[width=1.0\textwidth]{Logo-UTN-BA.jpeg}
 \end{minipage}	
 \begin{minipage}{7.7cm}
\fontsize{13.5pt}{12pt}\selectfont
%   \normalsize
   {
     \textsl 
     {
       \calligra{Universidad Tecnológica Nacional\\ Facultad Regional Buenos Aires \\ Departamento de {Ingeniería} {Electrónica}} 
     }
   }
 \end{minipage}
}

\fancyhead[R] %la "R" indica a la derecha
{
  \begin{minipage}{4.0cm}
   \small
   {
     \emph{\textbf{Informática I}} \\ \emph{18 de Julio de 2020} \\ \emph{Primer Parcial}\\ \emph{Curso R1021}  
   }
  \end{minipage}}

\definecolor{OliveGreen}{RGB}{2,80,1}
\definecolor{LigthOrange}{RGB}{255,255,200}
\definecolor{gray97}{gray}{.97}



\begin{document} % Inicio del documento
\newpage
%Cambia de página, el texto después de este comando aparecerá en la siguiente página en adelante.


\noindent
  \begin{center}
   \begin{tabular}{| c | c | c |}
    \hline
     Apellido y Nombres \hspace{8cm} &  Legajo & {Calificación} \\ \hline 
      &	& \\ \hline
   \end{tabular}	
  \end{center}

\noindent

%%%%%%%%%%%%%%%%%%%%%%%%%%%%%%%%%%%%%%%%%%%%%%%%%%%%%%%%%%%%%%%%%%%%%%%%%%%%%%%%%%%%%%%%%%%%%%%%%%%%%%%%%%%%%%%
%%%%%%%%%%%%%%%%%%%%%%%%%%%%%%%%%%%%%%%%%%%%%%%%%%%%%%%%%%%%%%%%%%%%%%%%%%%%%%%%%%%%%%%%%%%%%%%%%%%%%%%%%%%%%%%
% Comienzo del tema del examen
%%%%%%%%%%%%%%%%%%%%%%%%%%%%%%%%%%%%%%%%%%%%%%%%%%%%%%%%%%%%%%%%%%%%%%%%%%%%%%%%%%%%%%%%%%%%%%%%%%%%%%%%%%%%%%%
%%%%%%%%%%%%%%%%%%%%%%%%%%%%%%%%%%%%%%%%%%%%%%%%%%%%%%%%%%%%%%%%%%%%%%%%%%%%%%%%%%%%%%%%%%%%%%%%%%%%%%%%%%%%%%%

\lstset{
	frame=Ltb,
	framerule=0pt,
	aboveskip=0.5cm,
	framextopmargin=3pt,
	framexbottommargin=3pt,
	framexleftmargin=0.4cm,
	framesep=0pt,
	rulesep=.4pt,
	backgroundcolor=\color{gray97},
	rulesepcolor=\color{black},
 	language=C,
	captionpos=b,
	tabsize=3,
	frame=lines,
	keywordstyle=\color{blue},
	commentstyle=\color{Gray},
	stringstyle=\color{red},
	numbers=left,
	numberstyle=\tiny,
	numbersep=5pt,
	breaklines=true,
	showstringspaces=false,
	basicstyle=\ttfamily\scriptsize,
	emph={label},
	framerule=0pt,
}

\begin{enumerate}
\item Implemente una función que realice la validación de una CBU (Clave Bancaria Uniforme).\\
La CBU está formado de la siguiente manera.\\
\fontsize{9pt}{9pt}\selectfont
\vspace{-0.6cm}
\begin{center} 
   \begin{tabular}{|c|c|c|c|c|c|c|c|c|c|c|c|c|c|c|c|c|c|c|c|c|c|}
    \hline 
     2&9&9&0&0&7&7&2&0&7&7&1&1&6&9&5&5&7&0&0&1&1 \\ \hline 
     $E_0$&$E_1$&$E_2$&$S_0$&$S_1$&$S_2$&$S_3$&$D_0$&$C_0$&$C_1$&$C_2$&$C_3$&$C_4$&$C_5$&$C_6$&
     $C_7$&$C_8$&$C_9$&$C_{10}$&$C_{11}$&$C_{12}$&$D_1$\\ \hline 
   \end{tabular}
\end{center}   	
%\vspace{2.5cm}
\fontsize{12pt}{12pt}\selectfont
Donde:
\begin{itemize}
\item {\bf E0 a E2}: Es el número de la entidad bancaria.
\item {\bf S0 a S3}: Es el número de sucursal.
\item {\bf D0}: Es el dígito verificador de E y S
\item {\bf C0 a C12}: Es el número de cuenta.
\item {\bf D0}: Es el dígito verificador de C
\end{itemize}
Los dígitos verificadores de la clave bancaria única se calculan de la siguiente forma:\\
\fontsize{10.4pt}{10.4pt}\selectfont

$R_0$ = $E_0$*7 + $E_1$*1 + $E_2$*3 + $S_0$*9 + $S_1$*7 + $S_2$*1 + $S_3$*3 \\
$D_0$ = 10 - ($R_0$ \% 10)\\

$R_1$ = $C_0$*3 + $C_1$*9 + $C_2$*7 + $C_3$*1 + $C_4$*3 + $C_5$*9 + $C_6$*7 + $C_7$*1 + $C_8$*3 + $C_9$*9 + $C_{10}$*7 + $C_{11}$*1  + $C_{12}$*3\\
$D_1$ = 10 - ($R_1$ \% 10)\\
\fontsize{12pt}{12pt}\selectfont
\\
La función tiene el siguiente prototipo:
\begin{center}
{\bf int cbu\_validar (char *dataPtr)}
\end{center}
Donde {\bf dataPtr} es el puntero la clave bancaria uniforme a validar terminada en ‘{\bf $\setminus$0}’\\
\\
Devuelve:
\begin{itemize}
\item {\bf -1} Si la CBU pasada no tiene 22 caracteres.
\item {\bf -2} Si alguno de los caracteres de la CBU no es un número.
\item {\bf -3} Si el dígito verificador D0 no corresponde.
\item {\bf -4} Si el dígito verificador D1 no corresponde.
\end{itemize}
\end{enumerate}
 
\end{document} % Fin del documento.
 

  

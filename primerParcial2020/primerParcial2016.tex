\documentclass[12pt]{article} % Clase de documento: artículo y tamaño de letra
 
% \usepackage[spanish]{babel} % Manejo de idiomas
\usepackage[utf8]{inputenc} % Escritura en castellano con acentos
\usepackage[T1]{fontenc} % Escritura en castellano con acentos
\usepackage{calligra} 
\usepackage{listings}
\usepackage{pslatex}  % Fuente de letras
\usepackage{graphicx}
\usepackage{array}
%\usepackage{times} % Fuente de letras
\usepackage[margin=2.5cm]{geometry}
%Coloca 2.5 cm de margenes superior, inferior,derecho e izquierdo.
\usepackage{caption}
\usepackage{txfonts}
\usepackage{xcolor}
\usepackage{fancyhdr}
\usepackage[spanish,activeacute]{babel}
\usepackage{float}
\usepackage{multicol}
% \usepackage[usenames,dvipsnames]{color}
\usepackage{color}
% \usepackage[T1]{fontenc}
\usepackage{times}
\usepackage{tikz}
\usepackage{verbatim}
\usepackage{mdwlist}

\usetikzlibrary{chains,fit,shapes,arrows,calc,shapes,decorations.pathreplacing}
\usetikzlibrary{through}


\DeclareCaptionFont{white}{\color{white}}
\DeclareCaptionFormat{listing}{\colorbox[cmyk]{0.43, 0.35, 0.35,0.01}{\parbox{\textwidth}{\hspace{15pt}#1#2#3}}}

%Paquetes para hacer tablas lindas...
% \usepackage{amsmath,amssymb,amsfonts,latexsym,stmaryrd}
\usepackage{tabularx}
\usepackage{colortbl}
\usepackage{shadow}
\usepackage{fancybox}
\usepackage{url}
\usepackage[hidelinks]{hyperref}
\usepackage{subfigure}
\usepackage{multirow}


\definecolor{OliveGreen}{RGB}{2,80,1}
\definecolor{LigthOrange}{RGB}{255,255,200}
\definecolor{gray97}{gray}{.97}
\definecolor{Gray}{RGB}{171,174,178}


\pagestyle{fancy}
\headheight=50pt %para cambiar el tamaño del encabezado
\fancyhead[L]    %la "L" indica a la izquierda
{	
 \begin{minipage}{2cm}
  \includegraphics[width=0.6\textwidth]{UTNlogo.eps}
 \end{minipage}	
 \begin{minipage}{6.9cm}
   \normalsize
   {
     \textsf
     {
       \calligra{Universidad Tecnológica Nacional\\ Facultad Regional Buenos Aires \\ Departamento de {Ingeniería} {Electrónica}} 
     }
   }
 \end{minipage}
}

\fancyhead[R] %la "R" indica a la derecha
{
  \begin{minipage}{4.0cm}
   \small
   {
     \emph{\textbf{Informática I}} \\ \emph{3 de Julio de 2017} \\ \emph{Primer Parcial}\\ \emph{Curso r1092}  
   }
  \end{minipage}}

\definecolor{OliveGreen}{RGB}{2,80,1}
\definecolor{LigthOrange}{RGB}{255,255,200}
\definecolor{gray97}{gray}{.97}



\begin{document} % Inicio del documento
\newpage
%Cambia de página, el texto después de este comando aparecerá en la siguiente página en adelante.


\noindent
  \begin{center}
   \begin{tabular}{| c | c | c |}
    \hline
     Apellido y Nombres \hspace{8cm} &  Legajo & {Calificación} \\ \hline 
      &	& \\ \hline
   \end{tabular}	
  \end{center}

\noindent
Condiciones: \\
Bla
Bla
Bla{\bf primerParcial}.\\

%%%%%%%%%%%%%%%%%%%%%%%%%%%%%%%%%%%%%%%%%%%%%%%%%%%%%%%%%%%%%%%%%%%%%%%%%%%%%%%%%%%%%%%%%%%%%%%%%%%%%%%%%%%%%%%
%%%%%%%%%%%%%%%%%%%%%%%%%%%%%%%%%%%%%%%%%%%%%%%%%%%%%%%%%%%%%%%%%%%%%%%%%%%%%%%%%%%%%%%%%%%%%%%%%%%%%%%%%%%%%%%
% Comienzo del tema del examen
%%%%%%%%%%%%%%%%%%%%%%%%%%%%%%%%%%%%%%%%%%%%%%%%%%%%%%%%%%%%%%%%%%%%%%%%%%%%%%%%%%%%%%%%%%%%%%%%%%%%%%%%%%%%%%%
%%%%%%%%%%%%%%%%%%%%%%%%%%%%%%%%%%%%%%%%%%%%%%%%%%%%%%%%%%%%%%%%%%%%%%%%%%%%%%%%%%%%%%%%%%%%%%%%%%%%%%%%%%%%%%%

\lstset{
	frame=Ltb,
	framerule=0pt,
	aboveskip=0.5cm,
	framextopmargin=3pt,
	framexbottommargin=3pt,
	framexleftmargin=0.4cm,
	framesep=0pt,
	rulesep=.4pt,
	backgroundcolor=\color{gray97},
	rulesepcolor=\color{black},
% 	language=C,
	captionpos=b,
	tabsize=3,
	frame=lines,
	keywordstyle=\color{blue},
	commentstyle=\color{Gray},
	stringstyle=\color{red},
	numbers=left,
	numberstyle=\tiny,
	numbersep=5pt,
	breaklines=true,
	showstringspaces=false,
	basicstyle=\ttfamily\scriptsize,
	emph={label},
	framerule=0pt,
}

\begin{enumerate}
\item Implemente una función que contenga el algoritmo de cifrado {\bf ROT13}. \\
El algoritmo consiste en sustituir cada letra por una que se encuentra trece posiciones por delante. Por ejemplo la A se reemplaza por la N, la B por la O y así sucesivamente. Para las últimas trece letras la secuencia se invierte. A continuación se muestra la tabla de las equivalencias entre las letras.
%\vspace{0.6cm}
\noindent
  \begin{center}
   \begin{tabular}{| c | c | c | c | c | c | c | c | c | c | c | c | c |}
    \hline
     A & B & C & D & E & F & G & H & I & J & K & L & M \\ \hline 
     N & O & P & Q & R & S & T & U & V & W & X & Y & Z \\ \hline 
   \end{tabular}	
  \end{center}
Ejemplo:
\noindent
\vspace{-0.6cm}
  \begin{center}
   \begin{tabular}{| c | c | c | c | c | c | c | c | c | c | c |}
    \hline
     Texto sin cifrar & H & O & L & A &   & M & U & N & D & O  \\ \hline 
     Texto cifrado    & U & B & Y & N &   & Z & H & A & Q & B  \\ \hline 
   \end{tabular}	
  \end{center}
Prototipo de la función:
\begin{center}
{\bf int rot\_13 (char *dataPtr)}
\end{center}
Donde {\bf dataPtr} es el puntero al mensaje a cifrar (string) y donde se colocará el mensaje cifrado y devuelve:
\begin{itemize}
\item {\bf -1} Si el mensaje contiene un caracter {\bf distinto} de una {\bf letra mayúscula} o un {\bf espacio}.
\item Un {\bf número positivo} indicando la cantidad de caracteres convertidos sin tener en 
cuenta los espacios ni el {\bf $\setminus$0}
\end{itemize}
\item Implemente una función que realice la validación de una CBU (Clave Bancaria Uniforme).\\
La CBU está formado de la siguiente manera.\\
\fontsize{9pt}{9pt}\selectfont
\vspace{-0.6cm}
\begin{center} 
   \begin{tabular}{|c|c|c|c|c|c|c|c|c|c|c|c|c|c|c|c|c|c|c|c|c|c|}
    \hline 
     0&1&4&0&1&2&5&6&5&5&6&5&4&1&8&5&4&7&6&5&4&3 \\ \hline 
     $E_0$&$E_1$&$E_2$&$S_0$&$S_1$&$S_2$&$S_3$&$D_0$&$C_0$&$C_1$&$C_2$&$C_3$&$C_4$&$C_5$&$C_6$&
     $C_7$&$C_8$&$C_9$&$C_{10}$&$C_{11}$&$C_{12}$&$D_1$\\ \hline 
   \end{tabular}
\end{center}   	
%\vspace{2.5cm}
\fontsize{12pt}{12pt}\selectfont
Donde:
\begin{itemize}
\item {\bf E0 a E2}: Es el número de la entidad bancaria.
\item {\bf S0 a S3}: Es el número de sucursal.
\item {\bf D0}: Es el dígito verificador de E y S
\item {\bf C0 a C12}: Es el número de cuenta.
\item {\bf D0}: Es el dígito verificador de C
\end{itemize}
Los dígitos verificadores de la clave bancaria única se calculan de la siguiente forma:\\
\fontsize{10.4pt}{10.4pt}\selectfont

$R_0$ = $E_0$*9 + $E_1$*7 + $E_2$*1 + $S_0$*9 + $S_1$*7 + $S_1$*1 + $S_2$*3 \\
$D_0$ = 10 - ($R_0$ \% 10)\\

$R_1$ = $C_0$*9 + $C_1$*7 + $C_2$*1 + $C_3$*3 + $C_4$*9 + $C_5$*7 + $C_6$*1 + $C_7$*3 + $C_8$*9 + $C_9$*7 + $C_{10}$*1 + $C_{11}$*3  + $C_{12}$* 9\\
$D_1$ = 10 - ($R_1$ \% 10)\\
\fontsize{12pt}{12pt}\selectfont
\\
La función tiene el siguiente prototipo:
\begin{center}
{\bf int cbu\_validar (char *dataPtr)}
\end{center}
Donde {\bf dataPtr} es el puntero la clave bancaria uniforme a validar terminada en ‘{\bf $\setminus$0}’\\
\\
Devuelve:
\begin{itemize}
\item {\bf -1} Si la CBU pasada no tiene 22 caracteres.
\item {\bf -2} Si alguno de los caracteres de la CBU no es un número.
\item {\bf -3} Si el dígito verificador D0 no corresponde.
\item {\bf -4} Si el dígito verificador D1 no corresponde.
\end{itemize}
\item Implemente una función que obtenga el dígito verificador de un número de CUIT pasado como 
parámetro, el cálculo se realiza utilizando el algoritmo {\bf módulo11}.\\
 El prototipo de la función es el siguiente:
\begin{center}
{\bf int cuit\_validar (char *cuit);}
\end{center}
El parámetro {\bf cuit} es un puntero al vector que contiene el número de CUIT terminado en ‘{\bf $\setminus$0}’\\
\\
Devuelve:
\begin{itemize}
\item Un {\bf número positivo} indicando el dígito verificador. 
\item {\bf -1}: cuando la cantidad de dígitos es distinto de 10 
\item {\bf -2}: Indica que el número de CUIT es inválido (contiene algo distinto a números) \end{itemize}
Algoritmo módulo 11:  
\begin{itemize}
\item Multiplique los dígitos Desde el menos significativo por la serie 2,3,4,5,6,7.
\item Sume el resultado de las multiplicaciones anteriores.
\item Calcule el módulo 11 de la suma anterior.
\item Al resultado anterior restele 11
\begin{itemize}
\item si el resultado es {\bf menor que 10} lo obtenido es el dígito verificador
\item si el resultado es {\bf 10} el dígito verificador es {\bf 9}.
\item Si el resultado es {\bf 11} el dígito verificador es {\bf 0}.
\end{itemize} 
\end{itemize}
Ejemplo:\\
\\
%\vspace{0.3cm}
\noindent
\fontsize{10pt}{10pt}\selectfont
   \begin{tabular}{| c | c | c | c | c | c | c | c | c | c | c | c | c | c |}
    \hline
     CUIT & 2 & 0 & 1 & 2 & 3 & 4 & 5 & 6 & 7 & 8 & suma & \%11 & dígito\\ \hline 
     Valor a multiplicar por digito & $X_5$ & $X_4$ & $X_3$ & $X_2$ & $X_7$ & $X_6$ & $X_5$ & $X_4$ & $X_3$ & $X_2$ &  &148\%11 & 11-5 \\ \hline 
     Resultado de la multiplicación & 10 & 0 & 3 & 4 & 21 & 24 & 25 & 24 & 21 & 16 & =148 & =5 & 6\\ \hline 
   \end{tabular}	
\fontsize{12pt}{12pt}\selectfont
\item Explique qué entiende por variable, nombre los tipos que conozca y sus características. De ejemplos de uso.
\end{enumerate}
 
\end{document} % Fin del documento.
 

  
